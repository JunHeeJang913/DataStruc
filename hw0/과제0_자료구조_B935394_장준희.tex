\documentclass[a4paper]{article}

\usepackage{amsmath,amssymb}
\usepackage{txfonts}
\usepackage{kotex}
\usepackage{graphicx}
\usepackage{caption}
\usepackage{subfigure}
\usepackage{boxedminipage}
\usepackage{algorithm}
\usepackage{algpseudocode}
\begin{document}
\title{과제}
\author{B935394 컴퓨터공학과 장준희}
\date{2020 09 03}
\maketitle
\newpage
\section{자기소개}
이름:장준희\\학번:B935394\\학과:컴퓨터공학과\\
\section{수식 작성}
\subsection{학번별 수식 확인 후 옮겨적기}
\begin{boxedminipage}[l]{0.7\linewidth}
\# 19 \underline{\bf미분계수}
\begin{itemize}
\item[$\Diamondblack$]함수 $f(x)$의 $x=a$에서의 미분계수
\begin{itemize}
\item 함수 $y=f(x)$에서 $x$의 값이 $a$에서 $a+h$까지 변할 때,\\ $x$의 증분 $\Delta x\to 0$일 때 평균변화율의 극한값이 존재하면 이 극한값을 함수 $y=f(x)$의 $x=a$에서의 순간변화율 또는 미분계수라고 한다.
\item\begin{align*}f'(a)&=\lim_{\Delta x\to 0}\frac{\Delta y}{\Delta x}\\&=\lim_{h\to 0}\frac{f(a+h)-f(a)}{h}\\&=\lim_{x\to a}\frac{f(x)-f(a)}{x-a}\end{align*}
\end{itemize}
\item[$\Diamondblack$]미분계수의 기하학적의미
\begin{itemize}
\item 함수 $f(x)$의 $x=a$에서의 미분계수 $f'(a)$는 곡선 $y=f(x)$위의 점 $(a,f(a))$에서의 접선의 기울기를 나타낸다.
\end{itemize}
\item[$\Diamondblack$]미분가능성과 연속성
\begin{itemize}
\item 함수 $f(x)$의 $x=a$에서 미분가능하면 $f(x)$는 $x=a$에서 연속이다.\\
\end{itemize}
\end{itemize}
\end{boxedminipage}
\newline\\\\
글상자에서 미분계수의 정의와 이를 그래프로 옮겼을 때 볼 수 있는 기하학적 의미를 설명하고 있다.\\
\newpage
\section{가장 좋아하는 그림 첨부하기}
\subsection{자신을 나타낼 수 있는 사진}
\begin{figure}[h]
\begin{center}
\includegraphics[width=0.3\textwidth]{pizza}
\caption{피자}
\label{fig:fig1}
\end{center}
\end{figure}
그림 \ref{fig:fig1}은 피자이고 맛있다!!

\subsection{좋아하는 연예인 사진 3장}
\begin{figure}[h]
\subfigure[유재석/무한도전]{\includegraphics[width=0.3\textwidth]{yu1}}
\subfigure[유재석/런닝맨]{\includegraphics[width=0.3\textwidth]{yu2}}
\subfigure[유재석/미추리]{\includegraphics[width=0.3\textwidth]{yu3}}
\caption{유재석}
\label{fig:fig2}
\end{figure}
그림 \ref{fig:fig2}는 유재석의 다양한 프로 사진이다. 자기관리와 성실함을 배우고 싶다.
\newpage
\section{구구단}
\subsection{구구단 표출력}
\begin{center}
\begin{tabular}[h]{|c|c|c|c|}
\hline
2단&3단&4단&5단\\
\hline
$2\times1=2$&$3\times1=3$&$4\times1=4$&$5\times1=5$\\
\hline
$2\times2=4$&$3\times2=6$&$4\times2=8$&$5\times2=10$\\
\hline
$2\times3=6$&$3\times3=9$&$4\times3=12$&$5\times3=15$\\
\hline
$2\times4=8$&$3\times4=12$&$4\times4=16$&$5\times4=20$\\
\hline
\end{tabular}
\end{center}

구구단 표출력
\subsection{구구단 슈도코드}
\begin{algorithm}
	\caption{ Times\_Table }
	\begin{algorithmic}[1]
		\State $a, b$ is integar
		\For { $a=1,2, ... ,9$ }
			\For { $b=1,2, ... , 9$}
			\State Show $a \times b = a*b$
			\State $b = b+1$
			\EndFor
		\State newline
		\EndFor
	\end{algorithmic}
\end{algorithm}
			
	
\end{document}